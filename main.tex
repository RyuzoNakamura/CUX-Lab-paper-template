\documentclass{cuxarticle}
\begin{document}
% タイトル
\begin{titlepage}
  \begin{titlepage}
  \begin{center}
    {\large 20〇〇年度 学士学位論文}\\ % 年度
    \vspace{19\zh}
    {\Huge 論文主題}\\ % タイトル
    % {\Large ―サブタイトル―}\\ % サブタイトル(なければコメントアウト)
    \vspace{22\zh}
    \large{
      宮城大学 事業構想学群 価値創造デザイン学類\\
      〇〇デザインコース\\
      \vspace{1\zh}
      学籍番号\\
      氏名\\
      \vspace{3\zh}
      指導教員 佐藤 弘樹 准教授
    }
  \end{center}
\end{titlepage}

\end{titlepage}
% アブスト
\begin{center}
  {\Large
    論 文 要 旨\\
    \vspace{2\zh}
    タイトル
    \vspace{2\zh}
  }
\end{center}

%   ここにアブスト本文
{\normalfont
  「ダミーテキスト」DTPの普及前の組み版作業は「ダミーテキスト」近年、プロの作家やライターではない、一般の人たちが本を作ることが増えています。一昔前であればよほどの意気込みがなければ、「ダミーテキスト」印刷所に自分の原稿を持ち込み、自費で本を作る人はごく希でした。その時代、個人で方が本を作るには越えなければならない、いくつものハードルがありました。「ダミーテキスト」その一つが印刷費です。「ダミーテキスト」印刷物の宿命である版代が、「ダミーテキスト」部数にかかわらずかかってしまうことです。頁数が多く部数が少ない物ほど割高感が増し、個人を本作りから遠ざけてしまっていた大きな要因の一つではないかと思います。「ダミーテキスト」もう一つは、編集の作業です。
}

\vspace{3\zh}

\begin{flushright}
  宮城大学 事業構想学群 価値創造デザイン学類 〇〇デザインコース\\
  学籍番号\\
  氏名\\
\end{flushright}

% 目次
\tableofcontents

\chapter{序論}

\section{研究背景}
「ダミーテキスト」近年、プロの作家やライターではない、一般の人たちが本を作ることが増えています。一昔前であればよほどの意気込みがなければ、印刷所に自分の原稿を持ち込み、自費で本を作る人はごく希でした。その時代、個人で方が本を作るには越えなければならない、いくつものハードルがありました。その一つが印刷費です。印刷物の宿命である版代が、部数にかかわらずかかってしまうことです。頁数が多く部数が少ない物ほど割高感が増し、個人を本作りから遠ざけてしまっていた大きな要因の一つではないかと思います。「ダミーテキスト」もう一つは、編集の作業です。文章は書いても、それを体裁よく紙面に組むことは専門的な知識と経験をもったプロの仕事でした。著者が書いた原稿に、編集者がレイアウトや文字組の指示をつけ、それらにもとづいて組み版のオペレーターが紙面を作成していました。これらの作業にも当然、費用が掛かっていました。

\section{関連研究}

\subsection{小見出し}
「ダミーテキスト」それではなぜ現、一般の方にとって本作りが身近なものになってきたのでしょう。その一つにDTP(Desktop publishing)の発展と普及にあるといってよいでしょう。DTPとは、米アルダス社がページレイアウトソフト「ページメーカー」の発売に合わせ1986年に提唱した言葉です。新聞や書籍の編集(組み版)作業をPC上で行い、プリンターなどで紙面を出力するまでの工程を指します。DTPはここ10数年の間に飛躍的発展を遂げました。アルダス社に取って代わって、その旗手となったのが米アドビシステムズ社(以下アドビ社)です。アドビ社はDTPの主要ソフト「Adobe Photoshop」、「Adobe Illustrator」、「Adobe InDesign」の販売元であると同時に、現在のDTPにおける基幹的技術である「ポストスクリプト」(ページ記述言語)の開発元でもあります。事実上の「一社独占」と言って過言ではありません。その反面を置き、アドビ社がDTPの発展普及に大きく貢献してきたことは揺れもない事実です。

\subsection{小見出し2}
「ダミーテキスト」それではなぜ現、一般の方にとって本作りが身近なものになってきたのでしょう。その一つにDTP(Desktop publishing)の発展と普及にあるといってよいでしょう。DTPとは、米アルダス社がページレイアウトソフト「ページメーカー」の発売に合わせ1986年に提唱した言葉です。新聞や書籍の編集(組み版)作業をPC上で行い、プリンターなどで紙面を出力するまでの工程を指します。DTPはここ10数年の間に飛躍的発展を遂げました。アルダス社に取って代わって、その旗手となったのが米アドビシステムズ社(以下アドビ社)です。アドビ社はDTPの主要ソフト「Adobe Photoshop」、「Adobe Illustrator」、「Adobe InDesign」の販売元であると同時に、現在のDTPにおける基幹的技術である「ポストスクリプト」(ページ記述言語)の開発元でもあります。事実上の「一社独占」と言って過言ではありません。

\section{研究目的}
「ダミーテキスト」著者が書いた原稿に、編集者がレイアウトや文字組の指示をつけ、それらにもとづいて組み版のオペレーターが紙面を作成していました。これらの作業にも当然、費用が掛かっていました。新聞や書籍の編集(組み版)作業をPC上で行い、プリンターなどで紙面を出力するまでの工程を指します。DTPはここ10数年の間に飛躍的発展を遂げました。アルダス社に取って代わって、その旗手となったのが米アドビシステムズ社(以下アドビ社)です。アドビ社はDTPの主要ソフト「Adobe Photoshop」、「Adobe Illustrator」、「Adobe InDesign」の販売元であると同時に、現在のDTPにおける基幹的技術である「ポストスクリプト」(ページ記述言語)の開発元でもあります。事実上の「一社独占」と言って過言ではありません。

\section{本論文の構成}
本論文の構成は以下の通りである。

第1 章:序論
第2 章:関連研究
第3 章:計画
第4 章:実装
第5 章:考察と議論
第6 章:結論と今後の展望

第1 章では、本論文の研究背景や目的について記載した。
第2 章では、   について記載した。
第3 章では、   について記載した。
第4 章では、   について詳細に記載した。
第5 章では、   についてまとめた。
最後に第6 章では,本研究のまとめと,今後の展望を述べた。

\chapter{関連研究}
\section{中見出し}
「ダミーテキスト」それらにもとづいて組み版のオペレーターが紙面を作成していました。これらの作業にも当然、費用が掛かっていました。新聞や書籍の編集(組み版)作業をPC上で行い、プリンターなどで紙面を出力するまでの工程を指します。DTPはここ50数年の間に飛躍的発展を遂げました。アルダス社に取って代わって、その旗手となったのが米アドビシステムズ社(以下アドビ社)です。アドビ社はDTPの主要ソフト「Adobe Photoshop」、「Adobe Illustrator」、「Adobe InDesign」の販売元であると同時に、現在のDTPにおける基幹的技術である「ポストスクリプト」(ページ記述言語)の開発元でもあります。事実上の「一社独占」と言って過言ではありません。

\section{中見出し}
\subsection{小見出し}
「ダミーテキスト」アルダス社に取って代わって、その旗手となったのが米アドビシステムズ社(以下アドビ社)です。アドビ社はDTPの主要ソフト「Adobe Photoshop」、「Adobe Illustrator」、「Adobe InDesign」の販売元であると同時に、現在のDTPにおける基幹的技術である「ポストスクリプト」(ページ記述言語)の開発元でもあります。事実上の「一社独占」と言って過言ではありません。その反面を置き、アドビ社がDTPの発展普及に大きく貢献してきたことは揺れもない事実です。DTPの普及前の組み版作業はDTPの普及前の組み版作業は「ダミーテキスト」近年、プロの作家やライターではない、一般の人たちが本を作ることが増えています。

\subsection{小見出し}
「ダミーテキスト」新聞や書籍の編集(組み版)作業をPC上で行い、プリンターなどで紙面を出力するまでの工程を指します。DTPはここ十数年の間に飛躍的発展を遂げました。アルダス社に取って代わって、その旗手となったのが米アドビシステムズ社(以下アドビ社)です。アドビ社はDTPの主要ソフト「Adobe Photoshop」、「Adobe Illustrator」、「Adobe InDesign」の販売元であると同時に、現在のDTPにおける基幹的技術である「ポストスクリプト」(ページ記述言語)の開発元でもあります。事実上の「一社独占」と言って過言ではありません。

\chapter{計画}
\section{改ページ指定}
「ダミーテキスト」それらにもとづいて組み版のオペレーターが紙面を作成していました。これらの作業にも当然、費用が掛かっていました。新聞や書籍の編集(組み版)作業をPC上で行い、プリンターなどで紙面を出力するまでの工程を指します。DTPはここ十数年の間に飛躍的発展を遂げました。アルダス社に取って代わって、その旗手となったのが米アドビシステムズ社(以下アドビ社)です。アドビ社はDTPの主要ソフト「Adobe Photoshop」、「Adobe Illustrator」、「Adobe InDesign」の販売元であると同時に、現在のDTPにおける基幹的技術である「ポストスクリプト」(ページ記述言語)の開発元でもあります。事実上の「一社独占」と言って過言ではありません。

\subsection{小見出しダミーテキスト-1}
「ダミーテキスト」それらにもとづいて組み版のオペレーターが紙面を作成していました。これらの作業にも当然、費用が掛かっていました。新聞や書籍の編集(組み版)作業をPC上で行い、プリンターなどで紙面を出力するまでの工程を指します。DTPはここ10数年の間に飛躍的発展を遂げました。アルダス社に取って代わって、その旗手となったのが米アドビシステムズ社(以下アドビ社)です。

\subsection{小見出しダミーテキスト-2}
「ダミーテキスト」アドビ社はDTPの主要ソフト「Adobe Photoshop」、「Adobe Illustrator」、「Adobe InDesign」の販売元であると同時に、現在のDTPにおける基幹的技術である「ポストスクリプト」(ページ記述言語)の開発元でもあります。事実上の「一社独占」と言って過言ではありません。それらにもとづいて組み版のオペレーターが紙面を作成していました。これらの作業にも当然、費用が掛かっていました。

\subsection{小見出しダミーテキスト-3}
「ダミーテキスト」新聞や書籍の編集(組み版)作業をPC上で行い、プリンターなどで紙面を出力するまでの工程を指します。DTPはここ10数年の間に飛躍的発展を遂げました。アルダス社に取って代わって、その旗手となったのが米アドビシステムズ社(以下アドビ社)です。アドビ社はDTPの主要ソフト「Adobe Photoshop」、「Adobe Illustrator」、「Adobe InDesign」の販売元であると同時に、現在のDTPにおける基幹的技術である「ポストスクリプト」(ページ記述言語)の開発元でもあります。事実上の「一社独占」と言って過言ではありません。

\section{写真・図版の配置}
ランゲソー
アヤメ科アヤメ属 原産地:東アジア 開花期:6月 形態:多年草
陽地や湿地、水辺に自生。ランゲソー
アヤメ科アヤメ属 原産地:東アジア 開花期:6月 形態:多年草
陽地や湿地、水辺に自生。ランゲソー
アヤメ科アヤメ属 原産地:東アジア 開花期:6月 形態:多年草ランゲソー
アヤメ科アヤメ属 原産地:東アジア 開花期:6月 形態:多年草
陽地や湿地、水辺に自生。ランゲソー
アヤメ科アヤメ属 原産地:東アジア 開花期:6月 形態:多年草
陽地や湿地、水辺に自生。ランゲソー
アヤメ科アヤメ属 原産地:東アジア 開花期:6月 形態:多年草
陽地や湿地、水辺に自生。ランゲソー
アヤメ科アヤメ属 原産地:東アジア 開花期:6月 形態:多年草
陽地や湿地、水辺に自生。ランゲソー
アヤメ科アヤメ属 原産地:東アジア 開花期:6月 形態:多年草、水辺に自生。ランゲソー
アヤメ科アヤメ属 原産地:東アジア 開花期:6月 形態:多年草
陽地や湿地、水辺に自生。ランゲソー
アヤメ科アヤメ属 原産地:東アジア 開花期:6月 形態:多年草
陽地や湿地、水辺に自生。ランゲソー
アヤメ科アヤメ属 原産地:東アジア 開花期:6月 形態:多年草
陽地や湿地、水辺に自生。ランゲソー
アヤメ科アヤメ属 原産地:東アジア 開花期:6月 形態:多年草
陽地や湿地、水辺に自生。

\chapter{実装}
\section{改ページ指定}
「ダミーテキスト」それらにもとづいて組み版のオペレーターが紙面を作成していました。これらの作業にも当然、費用が掛かっていました。新聞や書籍の編集(組み版)作業をPC上で行い、プリンターなどで紙面を出力するまでの工程を指します。DTPはここ十数年の間に飛躍的発展を遂げました。アルダス社に取って代わって、その旗手となったのが米アドビシステムズ社(以下アドビ社)です。アドビ社はDTPの主要ソフト「Adobe Photoshop」、「Adobe Illustrator」、「Adobe InDesign」の販売元であると同時に、現在のDTPにおける基幹的技術である「ポストスクリプト」(ページ記述言語)の開発元でもあります。事実上の「一社独占」と言って過言ではありません。

\chapter{考察と議論}
\section{改ページ指定}
「ダミーテキスト」それらにもとづいて組み版のオペレーターが紙面を作成していました。これらの作業にも当然、費用が掛かっていました。新聞や書籍の編集(組み版)作業をPC上で行い、プリンターなどで紙面を出力するまでの工程を指します。DTPはここ十数年の間に飛躍的発展を遂げました。アルダス社に取って代わって、その旗手となったのが米アドビシステムズ社(以下アドビ社)です。アドビ社はDTPの主要ソフト「Adobe Photoshop」、「Adobe Illustrator」、「Adobe InDesign」の販売元であると同時に、現在のDTPにおける基幹的技術である「ポストスクリプト」(ページ記述言語)の開発元でもあります。事実上の「一社独占」と言って過言ではありません。

\chapter{結論と今後の展望}
\section{結論}
「ダミーテキスト」それらにもとづいて組み版のオペレーターが紙面を作成していました。これらの作業にも当然、費用が掛かっていました。新聞や書籍の編集(組み版)作業をPC上で行い、プリンターなどで紙面を出力するまでの工程を指します。DTPはここ十数年の間に飛躍的発展を遂げました。アルダス社に取って代わって、その旗手となったのが米アドビシステムズ社(以下アドビ社)です。アドビ社はDTPの主要ソフト「Adobe Photoshop」、「Adobe Illustrator」、「Adobe InDesign」の販売元であると同時に、現在のDTPにおける基幹的技術である「ポストスクリプト」(ページ記述言語)の開発元でもあります。事実上の「一社独占」と言って過言ではありません。

\section{今後の展望}
「ダミーテキスト」それらにもとづいて組み版のオペレーターが紙面を作成していました。これらの作業にも当然、費用が掛かっていました。新聞や書籍の編集(組み版)作業をPC上で行い、プリンターなどで紙面を出力するまでの工程を指します。DTPはここ十数年の間に飛躍的発展を遂げました。アルダス社に取って代わって、その旗手となったのが米アドビシステムズ社(以下アドビ社)です。アドビ社はDTPの主要ソフト「Adobe Photoshop」、「Adobe Illustrator」、「Adobe InDesign」の販売元であると同時に、現在のDTPにおける基幹的技術である「ポストスクリプト」(ページ記述言語)の開発元でもあります。事実上の「一社独占」と言って過言ではありません。

\subsection{小見出しダミーテキスト-1}
「ダミーテキスト」それらにもとづいて組み版のオペレーターが紙面を作成していました。これらの作業にも当然、費用が掛かっていました。新聞や書籍の編集(組み版)作業をPC上で行い、プリンターなどで紙面を出力するまでの工程を指します。DTPはここ10数年の間に飛躍的発展を遂げました。アルダス社に取って代わって、その旗手となったのが米アドビシステムズ社(以下アドビ社)です。

\subsection{小見出しダミーテキスト-2}
「ダミーテキスト」アドビ社はDTPの主要ソフト「Adobe Photoshop」、「Adobe Illustrator」、「Adobe InDesign」の販売元であると同時に、現在のDTPにおける基幹的技術である「ポストスクリプト」(ページ記述言語)の開発元でもあります。事実上の「一社独占」と言って過言ではありません。それらにもとづいて組み版のオペレーターが紙面を作成していました。これらの作業にも当然、費用が掛かっていました。

\subsection{小見出しダミーテキスト-3}
「ダミーテキスト」新聞や書籍の編集(組み版)作業をPC上で行い、プリンターなどで紙面を出力するまでの工程を指します。DTPはここ10数年の間に飛躍的発展を遂げました。アルダス社に取って代わって、その旗手となったのが米アドビシステムズ社(以下アドビ社)です。アドビ社はDTPの主要ソフト「Adobe Photoshop」、「Adobe Illustrator」、「Adobe InDesign」の販売元であると同時に、現在のDTPにおける基幹的技術である「ポストスクリプト」(ページ記述言語)の開発元でもあります。事実上の「一社独占」と言って過言ではありません。

\chapteraddtoc{謝辞}

本研究を進めるにあたり、   多くの方にご協力をいただきました。
本研究にあたり研究指導を行ってくださった   に深く感謝申し上げます。
本研究の実験に協力してくださった   に深く感謝申し上げます。
本研究全般に渡り多大なサポートをしてくださった   に深く感謝申し上げます。
本論文の執筆にあたり、多くのアドバイスをしてくださった   氏に深く感謝申し上げます。
など。好きに書いてOK。

20〇〇年〇月〇日
宮城大学 事業構想学群 価値創造デザイン学類 〇〇〇〇コース
氏名

\newpage
\renewcommand{\refname}{\huge 参考文献}
\bibliographystyle{junsrt}
\bibliography{ref}
\addcontentsline{toc}{chapter}{参考文献}

\end{document}
