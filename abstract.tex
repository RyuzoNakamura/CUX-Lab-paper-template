\begin{center}
  {\Large
    論 文 要 旨\\
    \vspace{2\zh}
    タイトル
    \vspace{2\zh}
  }
\end{center}

%   ここにアブスト本文
{\normalfont
  「ダミーテキスト」DTPの普及前の組み版作業は「ダミーテキスト」近年、プロの作家やライターではない、一般の人たちが本を作ることが増えています。一昔前であればよほどの意気込みがなければ、「ダミーテキスト」印刷所に自分の原稿を持ち込み、自費で本を作る人はごく希でした。その時代、個人で方が本を作るには越えなければならない、いくつものハードルがありました。「ダミーテキスト」その一つが印刷費です。「ダミーテキスト」印刷物の宿命である版代が、「ダミーテキスト」部数にかかわらずかかってしまうことです。頁数が多く部数が少ない物ほど割高感が増し、個人を本作りから遠ざけてしまっていた大きな要因の一つではないかと思います。「ダミーテキスト」もう一つは、編集の作業です。
}

\vspace{3\zh}

\begin{flushright}
  宮城大学 事業構想学群 価値創造デザイン学類 〇〇デザインコース\\
  学籍番号\\
  氏名\\
\end{flushright}
